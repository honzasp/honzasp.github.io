\documentclass[11pt,a4paper]{article}
\usepackage[utf8]{inputenc}
\usepackage{biolinum}
\usepackage{url}
\usepackage{hyperref}
\usepackage[top=10mm, bottom=15mm, left=15mm, right=15mm]{geometry}
\usepackage{parskip}
\usepackage{paralist}
\usepackage{microtype}
\usepackage{sectsty}
\usepackage{multicol}

\setlength{\columnsep}{8mm}
\setdefaultleftmargin{1.5em}{0em}{}{}{}{}
\allsectionsfont{\sffamily}
\hypersetup{hidelinks=true,colorlinks=true}

\begin{document}
\thispagestyle{empty}
\sffamily

{\bfseries\huge Jan Špaček}

\begin{multicols}{2}

I am Jan Špaček. My experience includes:
\begin{compactitem}
  \item High-performance distributed systems
  \item Deep learning
  \item Web applications
  \item Desktop applications
  \item Physically-based rendering
  \item Programming language implementations
\end{compactitem}

\section*{Work experience}

\textbf{Quantlane} (Prague, 2019)
  \newline Development and maintenance of a legacy trading system.
  \newline Python with asyncio, RabbitMQ

\textbf{Corona Renderer} (Prague, 2017–2019)
  \newline Integration of Corona Renderer into ARCHICAD.
  \newline Bachelor thesis supervised by Jaroslav Křivánek, defended in 2018.
  \newline Written in C++ for Windows, using ARCHICAD APIs and Corona API.

\textbf{Kiwi.com} (Brno, 2016, 2017)
  \newline Design and development of a custom distributed in-memory database
  with low latency and high throughput for storing flight combinations.
  \newline Development of an engine for distributed routing of flights.
  \newline Design and development of a distributed system for high-throughput
  reading of flights from Cassandra by directly reading the internal database
  representation.
  \newline C++, Python with asyncio, using Cassandra, Redis, PostgreSQL, and
  docker.

\textbf{Bileto} (Prague, 2015)
  \newline Development of an engine for real-time routing in public transport
    networks.
  \newline C++, using Redis and PostgreSQL.

\textbf{Adash} (Ostrava, 2014–2015)
  \newline Development of ADS, an application to visualize measured vibrations
  of industrial machinery.
  \newline Design of efficient digital filters accelerated using advanced
  features of ARM processors.
  \newline C++, wxWidgets.


\section*{Selected projects}

\textbf{SkyGAN} (2019–2020)
  \newline Generating high-resolution skydome images with deep learning
  (generative adversarial networks).
  \newline Master thesis supervised by David Futschik and Alexander Wilkie.
  \newline Python with PyTorch

\textbf{Dancerank.cz} (2016–)
  \newline A large database of dance sport results and competitions from several
  countries.
  \newline Analysis of results, ranking of couples, public API, ...
  \newline Predictions using a novel machine learning model.
  \newline Custom search engine in C++
  \newline Python with asyncio, MongoDB, Redis

\textbf{dort} (2016–2017)
  \newline A physically based renderer heavily influenced by pbrt.
  \newline C++, Lua

\textbf{spiral} (2015)
  \newline Implementation of a programming language: compiler, runtime support
  library with garbage collection, standard library.
  \newline C++, Rust, x86 assembler, Spiral.

\ldots{}and a large amount of smaller projects, a few of them are on my GitHub 
\href{https://github.com/honzasp}{@honzasp}.

\section*{Skills}

I am not limited to any particular language, platform or environment. I am most
experienced in C++ and Python, but I also wrote Rust,
{\small Haskell, C\#, Lua,}
{\footnotesize JavaScript,}
{\scriptsize Clojure, Ocaml, Go, Java, my own language Spiral, ...}

I happened to develop software mostly for Linux servers and Windows desktops. I
(ab)used many databases (MongoDB, PostgreSQL, Redis, Cassandra, ...), countless
libraries, frameworks, and APIs. I also enjoy low-level programming, such as
bare metal Beaglebone and Arduino, CUDA, or SIMD programming with SSE/AVX. My
editor is Vim.

I have solid background in computer science from Matfyz:
\begin{compactitem}
\item Bachelor (Bc.) (2015–2018, with honors): general computer science
\item Master (Mgr.) (2018–2020, with honors): artificial intelligence
\end{compactitem}

Beside computer science, I danced Latin on a competition level (with my partner
we have the highest national class ``A''). I am also an avid reader in English and
Czech.

\section*{Contacts}

Web: \url{https://honzasp.github.io}\newline
GitHub: \href{https://github.com/honzasp}{@honzasp}\newline
LinkedIn: \href{https://www.linkedin.com/in/jan-%C5%A1pa%C4%8Dek-3155b4136/}{Jan
Špaček}\newline

\end{multicols}
\end{document}
